%%
%% This is part of LFSread.
%%
%% $Author$
%% $Rev$
%% $Date::                           $
%%
\documentclass{lfsread}
\begin{document}

\LFStitle{2}{LFSブック日本語版のあれこれ}
\author{松 山 道 夫\thanks{\texttt{matsuand AT users DOT sourceforge DOT jp}}}
\date{2013年9月xx日}
\maketitle

\begin{abstract}
Linux From Scratch は Linux を一から作り出す手順書「LFSブック」を提供するプロジェクトです%
\footnote{\texttt{http://www.linuxfromscratch.org/}}。
著者はそのオリジナルに対する日本語訳を構築し、SourceForge サイト%
\footnote{\texttt{http://lfsbookja.sourceforge.jp/}}
において公開しています。
本書では、LFSブックのこと、著者の利用状況などについてや、
日本語訳を行う際の訳出術について、
また日本語版を生成するための処理段取りなどについて、取りとめもなく語ります。
\end{abstract}

\section{LFSブック日本語版の処理実装}

\subsection{オリジナル版の処理方式}

そもそもオリジナルのLFSブックは、DocBook XML DTD と表される書式に従った XML ファイル
がソースファイルとなっていて、これを処理することによって HTML ファイルや PDF ファイルが
生成されます。この XML ファイルには LFSブックの文章が記述されているわけですが、
XML ファイルがたった一つだけといった単純なものではありません。
各章をとりまとめるファイル、各章の中の個々のパッケージビルド手順を説明するファイル
といったものが、細かく分けられています。
したがって XML ファイルの総数は数十個にもなります。

\subsection{日本語訳を生み出すために考えること}

LFSブックの日本語訳を生み出すのに、まず第一に思いつくのは、オリジナルの XMLソースファイル
を直接書き換えて、英文で示される箇所を日本語に置き換えることです。
XMLソースファイルの英文の訳出をすべて終えたら、後はオリジナルの LFSブックと全く同様に
処理するだけで、日本語版 LFSブックが出来上がります。

ただこのやり方には多いに難点があります。
オリジナルのソースファイルが常に変化していくことに対して、
効果的に対処できないやり方であるという点です。
このことは実際にやってみないと、なかなか気づかないところかもしれません。



\end{document}
